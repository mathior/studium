%% Dokumentenklasse (Koma Script) -----------------------------------------
\documentclass[%
	final, % fertiges Dokument
	normalheadings, % keine grossen Ueberschriften wie es Standard waere
	ngerman, % wird an andere Pakete weitergereicht
	a4paper,
	1.1headlines, % Zeilenanzahl der Kopfzeilen
	pagesize, % Schreibt die Papiergroesse in die Datei.
	halfparskip, % Absaetze getrennt durch halbe Leerzeile, keine Einrueckung
	pointlessnumbers, % ueberschriftnummerierung ohne Punkt, siehe DUDEN
	fleqn, % Formeln werden linksbuendig angezeigt
]{scrartcl} % moegl. Klassen: scrartcl, scrreprt, scrbook
% -------------------------------------------------------------------------

\usepackage[utf8]{inputenc}
\usepackage[ngerman]{babel} % Languagesetting
%\usepackage{ragged2e} % Besserer Flatternsatz (Linksbuendig, statt Blocksatz)
\usepackage[]{amsmath} % Amsmath - Mathematik Basispaket
\usepackage[]{amssymb}
%\usepackage[T1]{fontenc} % T1 Schrift Encoding (notwendig für die meisten Type 1 Schriften)
%\usepackage{textcomp}	 % Zusätzliche Symbole (Text Companion font extension)
%\usepackage{lmodern}

\usepackage[%
%	dvips,
%	ps2pdf,
%	pdfpagelabels,
]{hyperref}

\hypersetup{
	breaklinks = {true}, % Links duerfen ueber mehrere Zeilen gehen
	linktocpage = {false}, % In TOC Seitenzahlen verlinken (true) oder Eintraege
	pdfborder = {0 0 0} % deaktiviert Border um Links
}

\begin{document}
\title{Diskrete Strukturen\\
       Hinweise zu den Übungen}
\author{Christian Forler, Christof Bräutigam, Michael Völske\\
    \texttt{<vorname>.<nachname>@uni-weimar.de}}
\date{\today}
\maketitle

\section{Einleitung}

Dieses Dokument ist eine Sammlung von Ideen, die uns beim Korrigieren der
Übungen zur Vorlesung \textit{Diskrete Strukturen} aufgefallen sind. Es
enthält Hinweise und Ratschläge allgemeiner Natur sowie Kritiken zu einzelnen
Aufgaben, oft auch Kriterien zur Punktevergabe. Bevor man sich über die
Bewertung einer Aufgabe beschwert, empfiehlt es sich also, hier nachzuschauen.

Das Dokument wird regelmäßig mit Einträgen zu den weiteren Übungsblättern
erweitert, wir bitten darum, ab und zu die neue Version zu ziehen.

Fragen, Kritik und Hinweise zu den Übungsaufgaben werden gern entgegengenommen.

\section{Allgemeine Hinweise}

\begin{itemize}
    \item Auf jedem Blatt bitte Name, Matrikelnummer und Studiengang
          vermerken.
    \item Bei Gruppenarbeit bitte nur ein Exemplar (mit allen Namen und
          Matrikelnummern der Gruppe) abgeben.
    \item Bitte die Blätter tackern oder binden
    \item Wenn es Fragen zu den Übungsaufgaben gibt, schickt einfach eine Mail.
          Das sollte aber möglichst nicht kurz vor der Abgabe geschehen.
\end{itemize}

Ein ganz wichtiger Punkt: \textbf{Lest die Aufgaben genau} und behandelt die
dort gestellten Probleme. Fehler im Aufgabenverständnis führen teilweise zu
Punktabzug. Einige Beispiele:

\begin{itemize}
    \item Ein induktiver Beweis über $ \mathbb{N}_0 $ muss auch mit $ n = 0 $
          im Induktionsanfang beginnen.
    \item Wenn $ a \Rightarrow b $ gezeigt werden soll, dann soll eben nicht
          $ b \Rightarrow a $ gezeigt werden und es ist auch nicht nötig
          $ a \Leftrightarrow b $ zu zeigen - letzteres ist zwar nicht falsch und
          gibt volle Punktzahl, es ist allerdings auch aufwendiger.
    \item Wenn eine Programmieraufgabe vorgibt, daß das Programm einen Parameter
          entgegennehmen soll und dazu ein Beispiel gegeben ist, dann soll es
          eben keine Eingabe vom Nutzer erwarten. Das führt zu weniger Code,
          damit zu weniger potentiellen Fehlerquellen und ermöglicht
          automatisierte Tests auf korrekte Funktion.
\end{itemize}

Ein Ratschlag: Es gibt viele Einzelabgaben, versucht bitte noch Gruppen zu bilden.
Das entlastet euch (allein sind die ersten Semester vom Arbeitsaufwand her kaum
zu schaffen, das ist uns bewusst), ist eine gute Vorbereitung auf spätere
Projektarbeit und spart auch Aufwand beim Korrigieren.

\section{Hinweise zu Quellcode-Abgaben}

Allgemein gilt: der Quellcode muss auf einem aktuellen frei verfügbaren System
kompilieren und das Programm i.d.R. mehrere Testcases bestehen.
Für \textbf{C/C++} benutzen wir den \textbf{gcc 4.4.3}, für \textbf{Java
javac 1.6.0\_18}. Wer nicht weiss, welche Version er im Einsatz hat, kann diese
Informationen mit \texttt{gcc -v} bzw. \texttt{javac -version} ermitteln.

Die Quellen werden auf der Kommandozeile kompiliert und ausgeführt, wer mit einer
IDE entwickelt achtet bitte darauf, daß der Code auch so funktioniert. Wir bewerten
neben der reinen Funktion auch die korrekte Umsetzung der Aufgabenstellung und
ziehen Punkte bei schweren stilistischen Mängeln ab.

\begin{itemize}
    \item Bitte Namen in einem Kommentar im Code mit angeben. Bei C/C++ gern
          auch Kürzel im Filenamen, das erspart das Umbenennen mehrerer
          ``\texttt{source.c}''-Files beim sammeln in einem Ordner.
    \item Bitte den Code nicht in die Mail schreiben sondern als Anhang senden,
          das erhält die Formatierung. Die Korrektur fällt leichter, wenn
          nicht erst editiert werden muss.
    \item Bitte sauber formatieren (konsequent einrücken) und wenn nötig
          kommentieren. Das erleichtert die Fehlersuche, wenn mal was nicht
          funktioniert und rettet vielleicht auch bei der Korrektur Punkte.
    \item Bitte keine fremden Libraries nutzen, keine Nicht-Standard-Funktionen
          und keine \texttt{system()}-Aufrufe. Das ist für die Bearbeitung der
          Aufgaben nicht nötig und verhindert im schlimmsten Falle, das der
          Quellcode compiled.
    \item Bitte nur Quellcodedateien im Anhang senden, keine Executables.
          (Es ist schlimm genug, daß die überhaupt durchs Mailsystem der Uni
          kommen. Abgesehen davon wird an der Professur für
          \textit{Mediensicherheit} niemand eine .exe aus einem Mailanhang
          ausführen.)
\end{itemize}

\subsection{C/C++ Quellcode}

\begin{itemize}
    \item Zur Formatierung: Welchen Einrückungsstil ihr bevorzugt bleibt euch
          überlassen aber bleibt bitte konsequent.
    \item Bitte auf Standardkompatibilität achten. Die Aufgabenstellungen
          erwähnen nicht umsonst ANSI C/C++. Gerade Visual-C und -C++ schlagen
          gern Funktionen vor, die nicht überall vorhanden sind. Am besten mit
          einem \texttt{gcc} und \texttt{-ansi} kompilieren.
% alles nötige includen!
\end{itemize}

\subsection{Java Quellcode}

\begin{itemize}
    \item Bitte die Konventionen einhalten, z.B. Klassennamen groß geschrieben,
          camelCase für Funktionen und Variablen usw. Bitte sauber Formatieren,
          das erleichtert das Lesen. Wer Eclipse nutzt kann mit
          \texttt{Ctrl+Shift+F} automatisch formatieren, andere Editoren und
          IDEs haben solche Funktionen auch.
    \item \texttt{Java.lang.*} muss nicht importiert werden. Imports lassen
          sich auch in allen IDEs automatisch verwalten (z.B. in Eclipse:
          \texttt{Strg+Shift+O}).
\end{itemize}


\section{Zu einzelnen Aufgaben}

Hier werden der Reihe nach Hinweise zur Bewertung einzelner Aufgaben gegeben und
häufige Fehler angesprochen.

\subsection{Übungsblatt 1}

\subsubsection{Aufgabe 1}

Induktive Beweise haben eine Formelle Struktur und sollten bitte auch so notiert
werden: Induktionsvorraussetzung, Induktionsanfang, Induktionsbehauptung,
Induktionsschritt. Ein strukturiertes Vorgehen hilft auch Fehler zu vermeiden.

Vollständige Induktion ohne Induktionsanfang ist formal falsch, die betreffenden
Lösungen müssten also streng genommen mit 0 Punkten bewertet werden. Diese eine
Zeile zu notieren dürfte zudem einen vertretbaren Aufwand darstellen. Wenn der
IA fehlte wurde ein Punkt abgezogen.

\subsubsection{Aufgabe 1b}

Sinn der vollständigen Induktion ist es, von einer Aussage die für $n$ gilt
auf eine Aussage über $n+1$ zu schliessen. Das wurde in keiner (!) der
abgegebenen Übungen gemacht, stattdessen wurde oft die Richtigkeit der
Aussage über $n+1$ bewiesen, was jedoch nicht der Aufgabenstellung entsprach.
Daher sind die betreffenden Lösungen streng genommen falsch. Wurde zumindest 
die formelle Struktur der vollständigen Induktion eingehalten (vor allem der
IA notiert) gab es auf die Teilaufgabe trotzdem noch einen Punkt.

Besonders auffällig war, das diese Teilaufgabe falsch angegangen wurde, während
bei Teilaufgabe 1a die Induktion richtig war.

\subsubsection{Aufgabe 2}

Eine Aussage lässt sich nicht beweisen indem man ein paar willkürlich gewählte
Werte einsetzt und zeigt, daß sie damit wahr ist. Es existieren nämlich
unendlich viele weitere Zahlen, die man auch einsetzten könnte.

\subsubsection{Aufgabe 3}

Diese Aufgabe wurde von einigen nicht bearbeitet. Da wären Punkte mit etwas
nachdenken ohne viel Aufwand zu holen gewesen.

\subsubsection{Aufgabe 3b}

Oft wurde richtig erkannt daß die Anzahl der Funktionen einer Variation mit
Zurücklegen entspricht, dann aber eine Variable $k$ eingeführt (die
typischerweise in Matheunterlagen an der Stelle verwendet wird), ohne
festzulegen, wie groß $k$ ist (nämlich $2^n$). Das gab einen Punkt Abzug.
$(2^n)^k$ ist als Lösung einfach falsch, auch wenn die Begründung stimmt (den
Punkt für die Begründung gab es trotzdem).

\subsection{Blatt 2}

\subsubsection{Aufgabe 4 allgemein}

Allgemein gilt: Bei der Rechnung in Restklassen geht es immer um ganze Zahlen.

Wir bitten auch um saubere Schrift bei Symbolen: $= und \equiv$ sollten klar
unterscheidbar sein, $\Rightarrow$ hat genau dieses und kein anderes Symbol usw.
Bei Rechenwegen bitten wir darum, auch Zwischenschritte (Gleichungen beidseitig
erweitern u.ä.) zu notieren, das hilft beim Nachvollziehen der Gedankengänge.

Bei Verweisen auf die Unterlagen bitte darauf achten ob auf eine Definition oder
einen Satz verwiesen wird (das ist nicht das Gleiche).

\subsubsection{Aufgabe 2a}
Wenn der Rechenweg falsch oder nicht nachvollziehbar war, gab es keinen Punkt,
auch wenn das Ergebnis stimmte.

% \subsubsection{Aufgabe 2.4.a}
% Def. 21 führt allgemein den Begriff der Kongruenz zw. zwei Zahlen a,b ein,
% lässt sich aber nicht heranziehen um die Eigenschaft der Reflexivität (Satz 22)
% zu beweisen.
% Stattdessen müsste Satz 23 angewendet werden (überführen aus der Restklasse in
% $\mathbb{N}$) und dort mit Hilfe von ``jede Zahl teilt die 0'' ($n|0 \forall n$,
% gilt immer als direkte Folgerung aus $a*0 = 0*a = 0$ in den Natürlichen Zahlen)
% gezeigt werden.

\subsubsection{Aufgabe 4.a-c}
Wenn keine Beweise geführt wurden (Rechenweg!) sondern nur auf den jeweiligen
Satz aus den Vorlesungsunterlagen verwiesen wurde, gibt es trotzdem den Punkt.
Es lohnt sich aber trotzdem etwas nachzudenken: Sätze sind keine Axiome und
lassen sich beweisen - und wenn das in der VL gezeigt wurde, sollte es umso
einfacher sein.

\subsubsection{Aufgabe 4.d-e}
Oft wurden keine direkten Beweise geführt (über Satz 23) sondern die
Rechenregeln (Satz 24) korrekt angewendet. Das ist Schade aber prinzipiell ok
und gab auch Punkte.

Andererseits wurde teilweise die Folgerung in die falsche Richtung gezeigt.
Bitte die Aufgaben gründlich lesen.

\subsubsection{Aufgabe 5}

Den Nutzer zur Eingabe einer Hexzahl aufzufordern ist kein Parameter für das
Programm. Gemeint war, den Wert als Kommandozeilenparameter (genauer: als
Argument) zu übergeben. Wenn das der einzige stilistische Mangel war, gab es
keinen Punktabzug.

Die Punkte wurden wie folgt verteilt: Wenn der Code nicht kompiliert gibt es
0 Punkte. 1 Punkt gab es, wenn das Beispiel aus der Aufgabenstellung das
richtige Ergebnis lieferte, als weiterer Testcase wurde die Quersumme von 
$123456789abcdef$ berechnet ($120$) um zu kontrollieren ob alle Zweige des
Programms richtig arbeiten. Weiterhin wurde die Nähe zur Aufgabenstellung
(Hexwert als Kommandozeilenparameter, Hexwert ist nicht in der Länge
beschränkt) und der Stil (sauberer Code, kein unnötiger, zu langer, zu
komplexer Code, kein Systemspezifischer Code) als Kriterium herangezogen.


\section{Der kleine Horrorladen}
\label{wtf}

Hier werden besonders kuriose, kreative (aber nichtsdestotrotz falsche) oder 
unsinnige Lösungen gezeigt. Dieser Abschnitt ist bitte nicht als Pranger zu
verstehen (es werden auch keine Namen genannt) sondern als Dokumentation von
Fehlern, die man besser unterlassen sollte - manchmal lohnt sich auch ein wenig
Nachdenken darüber, wie es zu dem Fehler kommen konnte oder warum die jeweilige
Sache eigentlich falsch ist.

\subsection{Rechenaufgaben}

In Restklassen wird nur mit ganzen Zahlen gerechnet. Wie man da auf Idee kommen
kann, $1/2$ in einer Rechnung einzusetzen, erschliesst sich uns nicht. Vielleicht
um eine falsche Aussage mit aller Macht doch als wahr zu beweisen...

Wer in der Lage ist, das Konzept des ``Teilers'' in $\mathbb{Q}$ zu erklären,
schickt bitte eine Mail an uns.

Wenn die Darstellung einer Zahl zur Basis $b$ Ziffern größer oder gleich $b$
enthält, sollte man sich Gedanken machen.

\subsection{Programmieraufgaben}

\texttt{goto} zum verzweigen in Unterprogramme gibt demnächst 0 Punkte!
Unsere Raketenwissenschaftler haben inzwischen High-Level-Programming-Features
(Verzweigungen, Schleifen, gar Funktionen) entwickelt, die sogar schon in den
ersten State-of-the-art-Programmiersprachen verfügbar sein sollen. Wir bitten
darum, sie zu nutzen.
Es gibt nur sehr wenige Fälle in denen \texttt{goto} sinnvoll ist und in den
Übungen tritt ein solcher Fall sicher nicht auf.

Uni-Übungen unterliegen keinem Copyright - schon gar nicht hierzulande und mit
falscher Schreibweise, nämlich ohne Angabe des Rechteinhabers. Ihr erhaltet
automatisch mit Schaffung eines Werkes das Urheberrecht daran, es bedarf
keiner expliziten Auszeichnung. Darüber hinaus sollte man sich Gedanken
machen, ob man mit dem, was man da mit seinem Namen versieht, wirklich in
Verbindung gebracht werden möchte.

Das Shell-Kommando \texttt{PAUSE} (C-Code: \texttt{system("PAUSE");}) wurde auf
keinem der Rechner der Korrektoren gefunden. Wenn sich eure Kommandozeile immer
gleich nach Beendigung des Programmes wieder schliesst, denkt mal darüber nach,
sie vielleicht vor dem Start zu öffnen und das Programm von dort auszuführen.

Ein .txt mit dem Quellcode drin zu schicken mag gerade noch ok sein (man kann
auch gleich .c oder .java als Suffix verwenden). Nicht ok ist, den Quellcode
in ein .doc zu packen. Das gibt in Zukunft 0 Punkte.

Sinnlose Kommentare (``\texttt{int quersumme; //klar, oder?}'') und
Stuss-Ausgaben (``dieses programm zerstört sich in wenigen sekunden selbst...
*kaboom*'') könnt ihr euch in Zukunft schenken, sonst fangen wir an die Abgaben
genau so ernst zu nehmen wie die betreffenden Leute ihre Aufgaben. Wir sind
ganz und gar keine Spaßbremsen, im Gegenteil, wir sehen gern coole Hacks und
Coderhumor aber Stumpfsinn ist Zeitverschwendung.

\end{document}

